\documentclass[12pt]{article}

\usepackage{amsmath,amssymb,enumerate,graphicx,pgf,tikz,fancyhdr}
\usepackage[shortlabels]{enumitem}
\usepackage[utf8]{inputenc}
\usetikzlibrary{arrows}
\usepackage{geometry}
\usepackage{tabvar}
\usepackage{bbm}

\geometry{hmargin=2.2cm,vmargin=2.5cm}\pagestyle{fancy}
\rfoot{\bfseries\thepage}
\cfoot{}
\renewcommand{\footrulewidth}{1.8pt}
\renewcommand{\headrulewidth}{1.8pt}
\input{macros}

\makeatletter
\patchcmd{\f@nch@head}{\rlap}{\color{primaryColor}\rlap}{}{}
\patchcmd{\headrule}{\hrule}{\color{primaryColor}\hrule}{}{}
\patchcmd{\f@nch@foot}{\rlap}{\color{primaryColor}\rlap}{}{}
\patchcmd{\footrule}{\hrule}{\color{primaryColor}\hrule}{}{}
\makeatother


\usepackage[mathscr]{euscript}
\usepackage{amsmath}

\begin{document} 

\begin{titlepage}
\lhead{\textcolor{primaryColor}{ML M1-IDD - 2020/2021}}

\newcommand{\HRule}{\rule{\linewidth}{0.5mm}} 

\center
 
%----------------------------------------------------------------------------------------
%	HEADING SECTIONS
%----------------------------------------------------------------------------------------


\includegraphics[scale=0.6]{upd.png}
\center\textsc{\large M1-I2D}\\[0.5cm]
\vspace{2cm}
\textsc{\Large \textcolor{primaryColor}{TD 4 :} Machine Learning}\\[1.5cm] 


%----------------------------------------------------------------------------------------
%	TITLE SECTION
%----------------------------------------------------------------------------------------

\HRule \\[0.4cm]
{ \huge \bfseries Generalized Linear Models}\\[0.4cm] 
\HRule \\[1.5cm]
 
%----------------------------------------------------------------------------------------
%	AUTHOR SECTION
%----------------------------------------------------------------------------------------

\begin{minipage}{0.4\textwidth}
\begin{flushleft} \large
Benamara Abdelkader Chihab 
\end{flushleft}
\end{minipage}
\vspace{2cm}

%----------------------------------------------------------------------------------------
%	DATE SECTION
%----------------------------------------------------------------------------------------


{\large February 13, 2021}\\[2cm]



\vfill 

\end{titlepage}

\Exo{1}{GLM \& exponential probability laws }
Soit la loi utiliséee pour $y_i$ est une loi provenant de la famille exponentielle :
\begin{equation}
    \Pb(Y=y_i) = b(y_i) \exp(\eta_iy_i-a(\eta_i))
    \label{eq:e1}
\end{equation}
Et on pose $\eta_i = \theta^Tx_i$ pour utiliser la loi dans un GLM.
\begin{enumerate}[1)]
    \item On pose $b(y) = 1$  et $a(\eta_i) = - \ln(1-\sigma(\eta_i))$ avec $\sigma(.)$ est la fonction sigmoide définit par $\sigma(z) = \frac{1}{1+\exp(-z)}$ et $\sigma^{-1}(p)=\ln(\frac{p}{1-p})$.   
     \\
     On posant $\mu_i = \sigma(\eta_i)$ alors on aura $\eta_i = \sigma^{-1}(\mu_i)$ remplaçons donc dans lo formule \eqref{eq:e1}
     $$
      \Pb(Y=y_i \lvert x_i,\theta ) = \exp(\eta_iy_i+\ln(1-\sigma(\eta_i)))
     $$
     $$
      = \exp(\sigma^{-1}(\mu_i)y_i+\ln(1-\sigma(\sigma^{-1}(\mu_i))))
     $$
     $$
      = \exp(\sigma^{-1}(\mu_i)y_i+\ln(1-\mu_i))
     $$
     $$
      = \exp(\ln(\frac{\mu_i}{1-\mu_i})y_i+\ln(1-\mu_i))
     $$
     Et donc :
     $$
     \Pb(Y=y_i \lvert x_i,\theta ) = \exp(y_i\ln(\mu_i) +(1-y_i)\ln(1-\mu_i))
     $$
     Et on a pour $y_i=1$ : 
     $$
     \Pb(Y=1 \lvert x_i,\theta ) = \exp(\ln(\mu_i)) = \mu_i = \sigma(\eta_i)=\sigma(\theta^Tx_i)
     $$
     \item \textbf{(Fonction de lien canonique)} Pour la régression logistique on a la fonction g définit par : $g(z) = \sigma^{-1}(z) = \ln(\frac{z}{1-z})$ et on a dans ce cas $\E(Y=y_i\lvert x_i,\theta) = \mu_i = \sigma(\eta_i)$ pour tout $i$.\\
     Dans le cadre de la régression logistique les $y_i$ suivent une loi Bernoulli de parametre $\mu_i$ leurs loi s'ecrivent pour $y\in \{0,1\}$: 
     $$
     \Pb(Y = y_i \lvert x_i,\theta ) =\mu_i^{y_i}(1-\mu_i)^{1-y_i} = \exp(\ln(\mu_i^{y_i}(1-\mu_i)^{1-y_i}))
     $$
     $$
     =\exp(y_i\ln(\mu_i)+(1-y_i)\ln((1-\mu_i)))
     $$
     D'apres la question (1) on a $a(\eta_i) = - \ln(1-\sigma(\eta_i))$ et donc : 
     $$
     a'(\eta_i)=\frac{\sigma'(\eta_i)}{1-\sigma(\eta_i)} = \frac{\sigma(\eta_i)(1-\sigma(\eta_i))}{1-\sigma(\eta_i)} = \sigma(\eta_i) = g^{-1}(\eta_i)
     $$
     Et donc on en déduit bien que la fonction de lien pour la régression logistique est canonique.
     
\end{enumerate}
\end{document}